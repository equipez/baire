\documentclass[11pt,a4paper]{article}
%\documentclass[11pt,a4paper,draft]{article}

\voffset=-1.5cm \hoffset=-1.4cm \textwidth=16cm \textheight=22.0cm

\usepackage{amsmath,amsthm,amssymb,amsfonts}
\usepackage{empheq}
\usepackage{bm}
\usepackage{enumerate}
\usepackage{lscape}
\usepackage{longtable}
\usepackage{color}
\usepackage[bbgreekl]{mathbbol}
\DeclareSymbolFontAlphabet{\mathbbm}{bbold}
\DeclareSymbolFontAlphabet{\mathbb}{AMSb}
\usepackage{bbm}
\usepackage{url}
\usepackage{rotating}
\usepackage{eqlist}

% graph, tikz and pgf
%\usepackage{subfigure}
\usepackage{graphicx}
\usepackage{tikz,tikzscale,pgf,pgfarrows,pgfnodes,filecontents,tikz-cd,}
\usetikzlibrary{arrows,arrows.meta,patterns,positioning,decorations.markings,shapes}
\usepackage{pgfplots}
\usepackage{pgfplotstable}
\usepackage[justification=centering]{caption}
\usepgfplotslibrary{fillbetween}
\pgfplotsset{compat=1.11}
%%%%%%%%%%%%%%%%%%%%%%%%%%%%%%%%%%%%%%%%%%%%%%%%%%%%%%%%%%%%%%%%%%%%%%

\usepackage{accents}
\usepackage{xspace}
\usepackage{silence}
\WarningFilter{latex}{Writing or overwriting file} % Mute the warning about 'writing/overwriting file'
\WarningFilter{latex}{Writing file} % Mute the warning about 'writing/overwriting file'
\WarningFilter{latex}{Tab has} % Mute the warning about 'Tab has been converted to Blank Space'
\usepackage[normalem]{ulem}
\usepackage[toc,page]{appendix}
\renewcommand{\appendixpagename}{\Large{Appendix}}
\renewcommand{\appendixname}{Appendix}
\renewcommand{\appendixtocname}{Appendix}
%\usepackage{sectsty}
\usepackage{multirow,booktabs}
\usepackage{enumitem}
\usepackage{upgreek}
\setlist[itemize]{leftmargin=*}

%\usepackage[nobottomtitles*]{titlesec} % No section title at the bottom of pages


%%%%%%%%%%%%%%%%%%%%%%%%%%%%%%%%%%%%%%
\definecolor{darkblue}{rgb}{0,0.1,0.5}
\definecolor{darkgreen}{rgb}{0,0.5,0.1}
\usepackage{hyperref}
\hypersetup{ colorlinks,%
linkcolor=darkblue,%
anchorcolor=darkblue,
citecolor=darkblue,%
urlcolor=darkblue}
\newcommand{\red}[1]{\textcolor{red}{#1}}
\newcommand{\blue}[1]{\textcolor{blue}{#1}}
%%%%%%%%%%%%%%%%%%%%%%%%%%%%%%%%%%%%%%

%\usepackage{refcheck} % Check unused labels

\usepackage[section]{algorithm}
\usepackage{algpseudocode,algorithmicx}
\newcommand{\INPUT}{\textbf{Input}}
\newcommand{\FOR}{\textbf{For}~}
\algrenewcommand\algorithmicrequire{\textbf{Input:}}
\algrenewcommand\algorithmicensure{\textbf{Output:}}
\algrenewcommand\alglinenumber[1]{\normalsize #1.}
\newcommand*\Let[2]{\State #1 $=$ #2}

\newtheorem{theorem}{Theorem}[section]
\newtheorem{acknowledgement}{Acknowledgement}[section]
\newtheorem{alg}{Algorithm}[section]
\newtheorem{axiom}{Axiom}[section]
\newtheorem{case}{Case}[section]
\newtheorem{claim}{Claim}[section]
\newtheorem{conclusion}{Conclusion}[section]
\newtheorem{condition}{Condition}[section]
\newtheorem{conjecture}{Conjecture}[section]
\newtheorem{corollary}{Corollary}[section]
\newtheorem{criterion}{Criterion}[section]
\newtheorem{exercise}{Exercise}[section]
\newtheorem{lemma}{Lemma}[section]
\newtheorem{notation}{Notation}[section]
\newtheorem{problem}{Problem}[section]
\newtheorem{proposition}{Proposition}[section]
\newtheorem{remark}{Remark}[section]
\newtheorem{solution}{Solution}[section]
\newtheorem{assumption}{Assumption}[section]
\newtheorem{summary}{Summary}[section]
\newtheorem{note}{Note}[section]
\newtheorem{doubt}{Doubt}[section]
\newtheorem{properties}{Properties}[section]
\newtheorem{example}{Example}[section]

\theoremstyle{definition}
\newtheorem{definition}{Definition}[section]

% Prevent footnote from running to the next page
\interfootnotelinepenalty=10000
% No line break in inline math
\interdisplaylinepenalty=10000
\relpenalty=10000
\binoppenalty=10000
% No widow or orphan lines
%\clubpenalty = 10000
%\widowpenalty = 10000
%\displaywidowpenalty = 10000

\usepackage{xpatch}
\xpatchcmd{\proof}{\itshape}{\normalfont\proofnamefont}{}{}
\newcommand{\proofnamefont}{\bfseries}

\usepackage{nccmath,relsize}
\DeclareMathOperator*{\mcap}{\mathsmaller{\bigcap}}
\DeclareMathOperator*{\mcup}{\mathsmaller{\bigcup}}
%\renewcommand{\cap}{\mathsmaller{\bigcap}}
%\renewcommand{\cap}{\mcap}

\setlength{\unitlength}{1mm}

\def\real{\mathbb{R}}
\def\SS{\mathbb{S}}
\def\ZZ{\mathbb{Z}}
\def\NN{\mathbb{N}}
\def\KK{\mathbb{K}}
\newcommand{\sss}[1]{{\scriptscriptstyle{#1}}}
\newcommand{\sups}[1]{{#1}}
\renewcommand{\top}{{\scriptstyle{\mathsf{T}}}}

\DeclareMathOperator{\sort}{sort}
\DeclareMathOperator*{\argmax}{argmax}
\DeclareMathOperator*{\argmin}{argmin}
\DeclareMathOperator{\Span}{span}
\DeclareMathOperator{\med}{med}
\newcommand{\co}{{\mathrm{c}}}
\newcommand{\rank}{\mathrm{rank}}
\newcommand{\range}{\mathrm{range}}
\newcommand{\ind}[2]{\operatorname{\mathbbm{1}}\;\!\!\!\big(#2\in#1\big)}
\newcommand{\diag}{\operatorname*{diag}}
\newcommand{\Diag}{\operatorname*{Diag}}
\newcommand{\im}{\operatorname{im}}
\newcommand{\diam}{\operatorname{diam}}
\newcommand{\dist}{\operatorname{dist}}
\newcommand{\Pred}{\mathrm{Pred}}
\newcommand{\cs}{\text{c}}
\newcommand{\hp}{\circ}
\newcommand{\card}{{\rm card}}
\newcommand{\fr}{\operatorname{fr}}
\newcommand{\sg}[1]{\bf { #1 }}
\newcommand{\ceil}[1]{ {\lceil{#1}\rceil} }
\newcommand{\floor}[1]{ {\lfloor{#1}\rfloor} }
%\renewcommand{\emph}{\textbf}
%\newcommand{\ones}{\mathbbm{1}}
\newcommand{\ones}{1}
\newcommand{\cc}{\sss{\textnormal{C}}}
\newcommand{\dec}{\sss{\textnormal{D}}}
\newcommand{\cauchy}{\sss{\textnormal{C}}}
\newcommand{\scauchy}{\sss{\textnormal{S}}}
\newcommand{\etc}{{etc.}}
\newcommand{\ie}{{i.e.}}
\newcommand{\eg}{{e.g.}}
\newcommand{\etal}{{et al.}}
\newcommand{\st}{\textnormal{s.t.}}
\newcommand{\me}{\mathrm{e}}
\newcommand{\md}{\mathrm{d}}
\newcommand{\lev}{\mathrm{lev}}
\newcommand{\bA}{\mathbf{A}}
\newcommand{\bx}{\mathbf{u}}
%\newcommand{\bb}{\mathbf{f}}
\newcommand{\bb}{\mathbf{r}}
\newcommand{\nov}{n_{\textnormal{o}}}
\newcommand{\MATLAB}{\textsc{Matlab}\xspace}
\newcommand{\rpss}{{SS}}
\newcommand{\rdfs}{{DF}}
\newcommand{\pfs}{{FS}}
\newcommand{\cg}{{CG}}
\newcommand{\hem}{{HEM}}
\newcommand{\prblm}{\texttt}
\DeclareMathAlphabet{\mathsfit}{T1}{\sfdefault}{\mddefault}{\sldefault}
\SetMathAlphabet{\mathsfit}{bold}{T1}{\sfdefault}{\bfdefault}{\sldefault}
\newcommand{\prbb}{\mathsfit{p}}
\newcommand{\pp}{\mathsf{p}}
\newcommand{\qq}{\mathsf{q}}
\newcommand{\ttt}{\mathsfit{t}}
\newcommand{\tol}{\varepsilon}
\newcommand{\bt}{\mathbf{t}}
\newcommand{\br}{\mathbf{r}}
\newcommand{\dd}{\mathbf{d}}
\newcommand{\ii}{\mathbf{i}}
\newcommand{\jj}{\mathbf{j}}
\newcommand{\xx}{\mathbf{x}}
\renewcommand{\pp}{\mathbf{p}}
\renewcommand{\ggg}{\mathbf{g}}
\newcommand{\GG}{\mathbf{G}}
\renewcommand{\Pr}{\mathbb{P}}
\newcommand{\iid}{\text{i.i.d.}}
\newcommand{\integer}{\textrm{I}}
\newcommand{\octave}{\mbox{GNU Octave}\xspace}
\newcommand{\gradp}{\nabla_{\!\sss{P_k}}}

% mathlcal font
\DeclareFontFamily{U}{dutchcal}{\skewchar\font=45 }
\DeclareFontShape{U}{dutchcal}{m}{n}{<-> s*[1.0] dutchcal-r}{}
\DeclareFontShape{U}{dutchcal}{b}{n}{<-> s*[1.0] dutchcal-b}{}
\DeclareMathAlphabet{\mathlcal}{U}{dutchcal}{m}{n}
\SetMathAlphabet{\mathlcal}{bold}{U}{dutchcal}{b}{n}

% mathscr font (supporting lowercase letters)
%\usepackage[scr=dutchcal]{mathalfa}
%\usepackage[scr=esstix]{mathalfa}
%\usepackage[scr=boondox]{mathalfa}
%\usepackage[scr=boondoxo]{mathalfa}
\usepackage[scr=boondoxupr]{mathalfa}
%\newcommand{\model}{\mathscr{h}}
\newcommand{\model}{\mathbf{h}}
\newcommand{\RR}{\mathbf{R}}
\newcommand{\TT}{\mathbf{T}}


\DeclareMathOperator{\inter}{int}
\DeclareMathOperator{\cl}{cl}
\DeclareMathOperator{\vol}{vol}
\newcommand{\Set}[1]{\mathcal{#1}}
\DeclareMathAlphabet{\mathpzc}{OT1}{pzc}{m}{it} % The mathpzc font
\newcommand{\slv}{\mathpzc} 
% mathpzc looks great, but it stops working on 19 Feb 2020 for no reason. 
%\newcommand{\slv}{\mathscr}
\newcommand{\software}{\texttt}
\DeclareMathOperator{\eff}{\mathsf{e}\;\!}
\DeclareMathOperator{\Eff}{\mathsf{E}\;\!}
\newcommand{\out}{{\text{out}}}

\numberwithin{equation}{section}
\setcounter{tocdepth}{2}

\DeclareMathOperator{\comp}{C}
\DeclareMathOperator{\sign}{sign}

\newcommand{\REPHRASE}[1]{{\color{blue}{#1}}}
\newcommand{\TYPO}[1]{{\color{orange}{#1}}}
\newcommand{\MISTAKE}[1]{{\color{violet}{#1}}}
\newcommand{\REVISION}[1]{{\color{blue}{#1}}}
\newcommand{\REVISIONred}[1]{{\color{red}{#1}}}
%\newcommand{\COMMENT}[1]{\textcolor{darkgreen}{(#1)}}
%\newcommand{\REVISION}[1]{#1}
%\newcommand{\REVISIONred}[1]{#1}
  

%%%%%%%%%%%%%%%%%%%%%%%%%%%%%%%%%%%%%%%%%%%%%%%%%%%%%%%%%%%%%%%%%%%%%
\title{Notes on Baire's Category Theory}
\date{November 14, 2020 (revised on \today)}
\author{
Z. Zhang
\thanks{
Department of Applied Mathematics, The Hong Kong Polytechnic
University, Hong Kong, China ({\tt zaikun.zhang@polyu.edu.hk}).
}
}

\begin{document}

\maketitle

\section{Basics}

\begin{definition}
  \label{def:intcl}
  Let~$X$ be a topological space and~$S$ be a set in~$X$. 
  \begin{enumerate}
    \item The interior of~$S$ is defined as~$\inter(S) = \bigcup\{U \mathrel{:} U\subset S \text{ and } U \text{ is open in } X\}$. 
    \item The closure of~$S$ is defined as~$\cl(S) = \bigcap\{G \mathrel{:} G\supset S \text{ and } G \text{ is closed in } X\}$. 
  \end{enumerate}
\end{definition}


\begin{proposition}
  \label{prop:intcl}
  Let~$A$ and~$B$ be two sets in a topological space~$X$. 
  \begin{enumerate}
    \item Duality between interior and closure: $\inter(A) = \cl(A^\co)^\co$, $\cl(A)=\inter(A^\co)^\co$.
    \item For any~$x\in X$, $x\in \inter(A)$ if and only if there exists an open set~$U$ such that~$x\in U \subset A$. 
    \item For any~$x\in X$, $x\in \cl(A)$ if and only if $U\cap A \neq \emptyset$ for any open set~$U$ such that~$x\in U$.
    \item $\inter(A\cap B) = \inter(A)\cap \inter(B)$.
      %, $\inter(A\cup B)\supset \inter(A)\cup \inter(B)$.
      %; if~$A$ is closed and~$\inter(B) = \emptyset$, then~$\inter(A\cup B) = \inter(A)$. 
    \item $\cl(A\cup B) = \cl(A)\cup \cl(B)$. 
      %, $\cl(A\cap B) \subset \cl(A)\cap \cl(B)$.
      %; if~$A$ is open and~$\cl(B) = X$, then~$\cl(A\cap B) = \cl(A)$.
    \end{enumerate}
\end{proposition}

\begin{proposition}
  \label{prop:subintcl}
  Let~$X$ be a topological space and~$Y$ be its subspace
  \begin{enumerate}
    \item For any~$S\subset Y$, $\inter_Y(S) \supset \inter_X(S)$;
      if~$Y$ is open in~$X$, then $\inter_Y(S)=\inter_X(S)$.
    \item For any~$S\subset Y$, $\cl_Y(S) = \cl_X(S) \cap Y$. 
  \end{enumerate}
\end{proposition}

In Proposition~\ref{prop:subintcl}, we use a subscript to indicate the topological space which the
interior or closure is defined with respect to. We will always do this if the context does not
suffice to avoid confusion. 

\begin{proof}
  1. $\inter_X(S)$ is open in~$X$ and~$\inter_X(S)\subset S\subset Y$. Thus~$\inter_X(S)$ is an open
     subset of~$S$ in~$Y$, leading to~$\inter_X(S)\subset \inter_Y(S)$. If~$Y$ is open,
     then~$\inter_Y(S)$ is an open subset of~$S$ in~$X$, implying that~$\inter_Y(S)\subset
     \inter_X(S)$ and hence~$\inter_Y(S) = \inter_X(S)$. 

  2. %First,~$\cl_X(S)\cap Y$ is a closed set in~$Y$ and it contains~$S$. 
     Since $\cl_Y(S)$ is closed in~$Y$, there exists a closed set~$G$ in~$X$ such
     that~$\cl_Y(S)= G\cap Y$. Thus~$S\subset  G$, implying that~$\cl_X(S) \subset G$.
     Therefore,~$\cl_X(S)\cap Y \subset G\cap Y = \cl_Y(S)$.  On the other hand,~$\cl_X(S)\cap Y$ is
     a closed set in~$Y$ and it contains~$S$, implying that~$\cl_X(S)\cap Y \supset \cl_Y(S)$.
     Hence~$\cl_X(S)\cap Y = \cl_Y(S)$.
\end{proof}

\begin{definition}
  \label{def:dense}
  %Let~$A$ and~$B$ be two sets in a topological space~$X$ such that~$A\subset B$. If~$\cl(A) \supset B$, then~$A$ is said to be dense in~$B$. 
  Let~$A$ and~$B$ be two sets in a topological space~$X$
  \begin{enumerate}
    \item If~$\cl(A) = X$, then~$A$ is said to be dense in~$X$. 
    \item If~$A\subset B$ and~$A$ is dense in~$B$ with respect to the subspace topology on $B$, 
    then we will simply say that~$A$ is dense in~$B$.  
  \end{enumerate}
\end{definition}

\begin{proposition}
  \label{prop:densein}
  Let~$A$ and~$B$ be two sets in a topological space~$X$ such that~$A\subset B$. Then~$A$ is dense
  in~$B$ if and only if~$\cl(A) \supset B$.
\end{proposition}

\begin{proof}
  $A$ is dense in~$B$ $\Leftrightarrow $ $\cl_B(A) = B$ $\Leftrightarrow$ $\cl(A)\cap B = B$
  $\Leftrightarrow$ $\cl(A) \supset B$. 
\end{proof}

\begin{proposition}
  \label{prop:dense}
  Let~$A$ and~$B$ be two sets in a topological space~$X$. 
  \begin{enumerate}
    \item $A\subset \cl(B)$ if and only if~$\inter(B^\co ) \subset \inter(A^\co)$.
    \item \label{it:abu} $A \subset \cl(B)$ if and only if~$U\cap B \neq \emptyset$ for any open
      set~$U$ such that~$U \cap  A \neq \emptyset$ 
      %(\ie, any neighborhood of any point in~$A$ has a nonempty intersection with~$B$).  
    \item \label{it:open} If~$A$ is open, then $A\subset \cl(B)$ if and only if $U\cap B \neq \emptyset$ for any
    nonempty open set~$U\subset A$.
  \item \label{it:abuu} If~$A\subset \cl(B)$, then $A\cap U \subset \cl(B\cap U)$ for any open set~$U$.
   %\item If~$A$ is dense in~$B$, then~$A\cap U$ is dense in~$B\cap U$ for any open set~$U$.  
   \item \label{it:dense}If $A$ is open and~$A\subset \cl(B)$, then~$A\cap B$ is dense in~$A$.
   %  , \ie, $A \subset \cl(A\cap B)$.
   \item \label{it:intcl} $\inter(\cl(A))\cap A$ is dense in~$\inter(\cl(A))$.
  \end{enumerate}
\end{proposition}

\begin{proof}
1. $A\subset \cl(B)$ $\Leftrightarrow$ $(\cl(B))^\co \subset A^\co $ 
$\Leftrightarrow$ $\inter(B^\co)  \subset A^\co$
$\Leftrightarrow$ $\inter(B^\co)  \subset \inter(A^\co)$.

2. Suppose that~$A\subset \cl(B)$. For any open set~$U$ such that~$U\cap A\neq \emptyset$, we
   have~$U\cap \cl(B)\neq \emptyset$, and hence~$U\cap B\neq \emptyset$. 
   If~$A\not\subset \cl(B)$, then~$U=(\cl(B))^\co$ is an open set such that~$U\cap A \neq \emptyset$
   while~$U\cap B = \emptyset$. 

3. Suppose that~$A\subset \cl(B)$. For any nonempty open set~$U\subset A$, we have~$U\cap A = U\neq \emptyset$, 
and hence~$U\cap B\neq \emptyset$ according to~\ref{it:abu}. 
   If~$A\not\subset \cl(B)$, then~$U=A\cap (\cl(B))^\co$ is a nonempty open set such that~$U\subset A$  
   while~$U\cap B = \emptyset$. 

4. Since~$A \subset \cl(B)$, for any open set~$V$ such that~$A\cap U\cap V\neq \emptyset$, we have
according to~\ref{it:open} that~$B\cap U\cap V\neq \emptyset$. This implies~$A\cap U \subset \cl(B\cap U)$
according to~\ref{it:open}.

5. According to~\ref{it:abuu}, $A = A\cap A \subset \cl(A\cap B)$ since~$A$ is open.

6. Since~$\inter(\cl(A))$ is open, and~$\inter(\cl(A)) \subset \cl(A)$, we know from 5 that 
$\inter(\cl(A))\cap A$ is dense in~$\inter(\cl(A))$. 
\end{proof}

\section{Cantor's theorem and its consequences}

\begin{theorem}[Cantor's theorem] 
  \label{th:cantor}
  Let~$X$ be a topological space, and~$\{C_n\}$ be a sequence of
  closed sets such that~$C_{n+1}\subset C_n$ for each~$n\ge 1$. 
\begin{enumerate}
  \item If~$X$ is complete metric space and~$\diam(C_n)\to 0$, then~$\bigcap_{n=1}^\infty C_n$ is
    a singleton.
  \item If each~$C_n$ is compact, then~$\bigcap_{n=1}^\infty C_n$ is nonempty.
\end{enumerate}
\end{theorem}


\begin{theorem}[{\cite[Theorems~1.4.5--1.4.6]{Zalinescu_2002}}]
  \label{th:intclcm}
  Let~$X$ be a complete metric space and $\{S_n\}_{n=1}^\infty$ be a sequence of sets in~$X$. 
  \begin{enumerate}
    \item If each~$S_n$ is open, then $\cl(\bigcap_{n=1}^\infty S_n)$ and~$\bigcap_{n=1}^\infty \cl(S_n)$ have the same interior. 
%      that is $\inter (\cl(\bigcap_{n=1}^\infty S_n)) = \inter(\bigcap_{n=1}^\infty \cl(S_n))$. 
    \item If each~$S_n$ is closed, then $\inter(\bigcup_{n=1}^\infty S_n)$ and~$\bigcup_{n=1}^\infty \inter(S_n)$ have the same closure. 
%      that is  $\cl (\inter(\bigcup_{n=1}^\infty S_n)) = \cl(\bigcup_{n=1}^\infty \inter(S_n))$.
  \end{enumerate}
\end{theorem}

\begin{proof}
  Due to the duality between closure and interior, we only prove~$1$, for which it suffices to show that 
  $\inter(\bigcap_{n=1}^\infty \cl(S_n))\subset \cl(\bigcap_{n=1}^\infty S_n)$.
  By item~\ref{it:open} of Proposition~\ref{prop:dense}, we only need to prove for a given
  nonempty open set~$U \subset \bigcap_{n=1}^\infty \cl(S_n)$ that%
  \begin{equation}
    \label{eq:usnonempty}
    U\mcap\left(\mcap_{n=1}^\infty S_n\right) \neq \emptyset. 
  \end{equation}
  To this end, we will define a sequence of closed balls~$\{B_n\}_{n=0}^\infty$~such~that% 
  \begin{equation}
    \label{eq:nest}
  0< \diam(B_n)< 2^{-n},
  \quad 
  B_{n+1} \subset B_{n} \subset  U\mcap\left(\mcap_{k=1}^n S_k\right)
  \quad \text{ for each } \quad  n \ge 0, 
  \end{equation}
  where $\bigcap_{k=1}^0 S_k=X$. 
  We obtain~\eqref{eq:usnonempty} once this is done, as
  Cantor's theorem will yield
    \begin{equation}
      \label{eq:nestnonempty}
      \emptyset \;\neq \;\mcap_{n=1}^\infty B_n \;\subset\; U\mcap \left(\mcap_{n=1}^\infty
      S_n\right).
    \end{equation}

    We define $\{B_n\}$ inductively. 
    As~$U$ is a nonempty open set, we can take a closed ball~$B_0\subset U$ 
    such that~$0<\diam(B_0)< 1$.
    Assume that the closed ball~$B_n$ is already defined for an~$n\ge 0$ 
    so that~$0<\diam(B_n)< 2^{-n}$ and~$B_n\subset U\cap(\bigcap_{k=1}^n S_k)$.
    Recalling that~$U\subset\cl(S_{n+1})$, 
    we have~$\inter(B_n)\subset \cl(S_{n+1})$, which implies that~$\inter(B_n) \cap S_{n+1} \neq
    \emptyset$. Since~$\inter(B_n)\cap S_{n+1}$ is open, we  
    can take a closed ball~$B_{n+1}\subset \inter(B_n)\cap S_{n+1}$ such that~$0<\diam(B_{n+1})< 2^{-(n+1)}$.  
    It is easy to see that~$B_{n+1}\subset B_n$ and 
    %$B_{n+1}\subset B_{n}\cap S_{n+1} \subset [U\cap (\bigcap_{k=1}^n S_{k})]\cap S_{n+1} = U\cap(\bigcap_{k=1}^{n+1} S_k)$. 
    $B_{n+1}\subset B_{n}\cap S_{n+1} \subset U\cap(\bigcap_{k=1}^{n+1} S_k)$. 
     This finishes the induction and completes the proof.
\end{proof}

\begin{theorem}
  \label{th:intcllch} Theorem~\ref{th:intclcm} still holds if~$X$ is a locally compact Hausdorff
  space. 
\end{theorem}

\begin{proof}
  The proof duplicates that of Theorem~\ref{th:intclcm}, except that~$\{B_n\}$ is now a sequence of
  compact sets with nonempty interior such that 
  $B_{n+1} \subset B_n \subset U \cap (\bigcap_{k=1}^n S_k)$, the existence of which can be
 established by the local compactness of~$X$.  Since~$X$ is a Hausdorff space, each~$B_n$ is closed,
 and hence~$\bigcap_{n=1}^\infty B_n$ is nonempty by Cantor's theorem. 
\end{proof}

\section{Baire's category theorem}

\begin{definition}
Let~$X$ be a topological space. 
\begin{enumerate}
  \item A set in~$X$ is said to be rare (or \emph{nowhere dense}) if its closure has empty interior.  
  \item A set in~$X$ is said to be meager if it is a countable union of rare sets.
  \item The complement of a meager set is called a comeager (or \emph{residual}) set.
  \item A meager set is also said to be of the first category; other sets are  
    of the second category.
\end{enumerate}
\end{definition}


\begin{proposition}
 Rare sets, subsets of meager sets, and countable unions of meager sets are all meager sets. 
\end{proposition}


\begin{definition}
  A topological space~$X$ is called a Baire space if every meager set in~$X$ has empty interior.
\end{definition}

\begin{proposition} 
   Let~$X$ be a topological space. The following statements are equivalent.  
  \begin{enumerate}
    \item $X$ is a Baire space.
    \item Any comeager set in~$X$ is dense.
    \item Any nonempty open set in~$X$ is not meager.
    \item Any countable intersection of dense open sets in~$X$ is still dense.
    \item Any countable union of rare closed sets in~$X$ has empty interior.
\end{enumerate}
\end{proposition}

\begin{proof}
  1 $\Rightarrow$ 2. 
  Let~$S$ be a comeager set in~$X$. Then~$S^\co$ is a meager
  set in~$X$. Thus~$S$ has empty interior, which implies that~$S$ is dense in~$X$. 

  2 $\Rightarrow$ 3. 
  Let~$U$ be an open set in~$X$.
  If~$U$ is meager, then~$U^\co$ is dense in~$X$. Thus~$U=\!\inter(U) \!= \emptyset$.

  3 $\Rightarrow$ 4. 
  Let~$\{U_n\}$ be a sequence of dense open sets in~$X$. Then~$(\bigcap_{n=1}^\infty U_n)^\co$ is
  meager. Its interior is an open meager set in~$X$, and hence empty. Thus~$\bigcap_{n=1}^\infty
  U_n$  is dense. 

  4 $\Rightarrow$ 5. 
  Let~$\{C_n\}$ be a sequence of rare closed sets in~$X$. Then~$\{C_n^\co\}$ is a seqeunce
  of dense open sets. Thus~$\bigcap_{n=1}^\infty C_n^\co$ is dense in~$X$. Taking the complement, we
  know that~$\bigcup_{n=1}^\infty C_n$ has empty interior.   

  5 $\Rightarrow$ 1. 
  Let~$S$ be a meager set in~$X$. Then there exists a sequence of rare sets~$\{S_n\}$ such that 
  $S=\bigcup_{n=1}^\infty S_n$. Thus~$S\subset \bigcup_{n=1}^\infty \cl(S_n)$, the latter of which has
  empty interior because~$\{\cl(S_n)\}$ is a sequence of rare closed sets. 
  Hence~$S$ has empty interior. 
\end{proof}

\begin{proposition}
  \label{prop:intclr}
  Let~$X$ be a topological space and~$S$ be a set in~$X$. Then~$S$ is rare in~$X$ if and only
  if~$\inter(\cl(S)) \cap S = \emptyset$.
\end{proposition}

\begin{proof}
  By item~\ref{it:intcl} of Proposition~\ref{prop:dense}, $\inter(\cl(S)) \cap S$ is dense
  in~$\inter(\cl(S))$. Thus $\inter(\cl(S)) = \emptyset$ if and only if $\inter(\cl(S)) \cap S = \emptyset$.
\end{proof}

\begin{lemma}
  \label{lem:openb}
  Let~$X$ be a topological space, $Y$ be its open subspace, and~$S$ be a subset of~$Y$. 
  \begin{enumerate}
    \item \label{it:rare}$S$ is rare in $X$ if and only if~$S$ is rare in~$Y$.
    \item $S$ is meager in $X$ if and only if~$S$ is meager in~$Y$.
  \end{enumerate}
\end{lemma}

\begin{proof} 
  1. Since~$Y$ is open in~$X$, we have
 \begin{equation}
   \label{eq:subrare}
     \inter_Y(\cl_Y(S)) = \inter_X(\cl_Y(S)) \;=\; \inter_X(\cl_X(S)\cap Y) = \inter_X(\cl_X(S))\cap Y.
 \end{equation}
   If~$S$ is rare in~$X$, then~$\inter_X(\cl_X(S))=\emptyset$, and hence~$\inter_Y(\cl_Y(S))
   = \emptyset$ by~\eqref{eq:subrare}, 
   implying that~$S$ is rare in~$Y$. 
   If~$S$ is rare in~$Y$, then~\eqref{eq:subrare} leads
   to~$\inter_{X}(\cl_X(S))\cap Y = \emptyset$, which implies~$\inter_X(\cl_X(S))\cap
   S = \emptyset$,  ensuring that~$S$ is rare in~$X$ by Proposition~\ref{prop:intclr}.

   2. If~$S$ is meager in~$X$, then~$S= \bigcup_{n=1}^\infty S_n$ with a sequence of rare
   sets~$\{S_n\}$ in~$X$. Each~$S_n$ is a subset of~$S$, and hence of~$Y$, and it is rare
   in~$Y$ according to~\ref{it:rare}. Thus~$S$ is meager in~$Y$. If~$S$ is meager in~$Y$, then~$S
   = \bigcup_{n=1}^\infty S_n$ with a sequence of rare sets~$\{S_n\}$ in~$Y$. Each~$S_n$ is rare in
   in~$X$ according to~\ref{it:rare}. Thus~$S$ is meager in~$X$. 
\end{proof}


\begin{theorem}
  \label{th:intcl}
   Let~$X$ be a topological space. The following statements are equivalent.  
   \begin{enumerate}
     \item $X$ is a Baire space.
     \item Any open subspace of~$X$ is a Baire space. 
     \item For any sequence~$\{S_n\}$ of open sets in~$X$, $\cl(\bigcap_{n=1}^\infty S_n)$ and
       $\bigcap_{n=1}^\infty \cl(S_n)$ have the same interior. 
     \item For any sequence~$\{S_n\}$ of closed sets in~$X$, $\inter(\bigcup_{n=1}^\infty S_n)$ and
       $\bigcup_{n=1}^\infty \inter(S_n)$ have the same closure. 
   \end{enumerate}
\end{theorem}

\begin{proof}
  1 $\Rightarrow$ 2. Let~$Y$ be an open subspace of~$X$, and~$S$ be a meager set in~$Y$.
  Then~$S$ is meager in~$X$ by Lemma~\ref{lem:openb}. Since~$X$ is a Baire space,~$S$ has empty interior in~$X$, which
  implies that~$S$ has empty interior in~$Y$. Hence~$Y$ is a Baire space. 

  2 $\Rightarrow$ 3. It suffices to prove that~$\inter(\bigcap_{n=1}^\infty \cl(S_n))\subset\cl(\bigcap_{n=1}^\infty S_n)$. 
  Let~$Y=\inter(\bigcap_{n=1}^\infty \cl(S_n))$. Then~$Y$ is a Baire space. 
  Define $T_n = S_n\cap Y$ for each~$n\ge 1$. Then each~$T_n$ is open in~$Y$. Since~$Y$ is open
  and~$Y\subset \cl(S_n)$, we know from item~\ref{it:dense} of Proposition~\ref{prop:dense}
  that~$T_n$ is dense in~$Y$. Hence~$\bigcap_{n=1}^\infty T_n$
  is dense in~$Y$. Therefore, $Y \subset \cl(\bigcap_{n=1}^\infty T_n) \subset
  \cl(\bigcap_{n=1}^\infty S_n)$ as desired. 

  3 $\Rightarrow$ 4. Obvious by the duality between interior and closure. 

  4 $\Rightarrow$ 5. Let~$\{S_n\}$ be a sequence of closed sets in~$X$ with empty interior. Then we
    have
  $\cl(\inter(\bigcup_{n=1}^\infty S_n)) = \cl(\bigcup_{n=1}^\infty \inter(S_n)) = \emptyset$.
  Thus~$\inter(\bigcup_{n=1}^\infty S_n) = \emptyset$. Hence~$X$ is a Baire space. 
\end{proof}

Theorems~\ref{th:intclcm}, \ref{th:intcllch}, and~\ref{th:intcl} lead us to Baire's Category
Theorem.

\begin{theorem}[Baire's Category Theorem, {\cite[Theorem~48.2]{Munkres_2000}}]
  \label{th:Baire}
  Complete metric spaces and locally compact Hausdorff spaces are Baire spaces.
\end{theorem}


\section{Baire's category in topological vector spaces}

\begin{definition}
 A topological vector space $X$ is a vector space over a topological field $\KK$ (most often the
 real or complex numbers with their standard topologies) that is endowed with a topology such that
 the vector addition $(x,y)\mapsto x+y$ and scalar multiplication $(\lambda, x) \mapsto \lambda x$ are continuous functions, where the domains of these
 functions are endowed with product topologies.
\end{definition}

\begin{proposition}
  Suppose that~$X$ be a topological vector space and~$U$ is a neighborhood of~$0$.
  Then~$X=\bigcup_{n=1}^\infty (nU)$. 
\end{proposition}

\begin{proof}
  For any~$x\in X$, $n^{-1} x\to 0$ due to the continuity of the scalar multiplication. 
  Since~$U$ is a neighborhood of~$0$,
  $n^{-1} x$ falls inside~$U$ when~$n$ is sufficiently large, which implies~$x\in n U$.
\end{proof}

\begin{theorem}[{\cite[Theorem~11.6.7]{Narici_Beckenstein_2010}}]
  If~$X$ is a topological vector space, then~$X$ is a Baire space if and only if it is not meager. 
\end{theorem}

\begin{proof}
 If~$X$ is a Baire space, then it is not meager as an open subset of itself.  If~$X$ is not a Baire
 space, then there exists a nonempty open set~$U$ that is meager. Take~$x\in U$ and set~$V=U-x$.
 Then~$V$ is a meager neighborhood of~$0$. Thus~$X=\bigcup_{n=1}^\infty (nV)$ is meager. 
\end{proof}

\section{Examples}

\begin{proposition}
  Let~$f\mathrel{:} (0,+\infty)\to \real$ be a continuous function. If~$\lim_{n\to \infty}f(nx) = 0$ 
   for all~$x>0$, then~$\lim_{x\to +\infty} f(x) = 0$.
\end{proposition}

\begin{proof} We give two proofs. 

  1. Proof by Cantor's theorem. Assume that~$f(x)\nrightarrow 0$ when~$x\to +\infty$. Then there
     exists a constant~$\epsilon > 0$ such that there exists~$x \ge M$ with~$|f(x)|>\epsilon$
     for any~$M > 0$. 
     We will define 
     a strictly increasing sequence of integers~$\{n_k\}$ and
     a decreasing nested sequence of closed intervals~$\{[a_k, b_k]\}$
     such that~$b_k>a_k > 0$ and 
     \begin{equation}
       |f(n_kx)| > \epsilon \quad \text{for all}\quad x \in [a_k, b_k] \quad \text{and}\quad   k\ge 1.
     \end{equation}
     Once this is done, then there exists~$y\in\bigcap_{k=1}^\infty [a_k, b_k]$, and~$f(n_ky)> \epsilon$ for each~$k\ge 1$, contradicting the assumption. 

     We define the aforementioned~$\{[a_k,b_k]\}$ and~$\{n_k\}$ inductively.
     Let~$n_1=1$. By the continuity of~$f$, there exists an interval~$[a_1, b_1]$ such
     that~$b_1>a_1>0$ and~$f(n_1x)> \epsilon$ for all~$x\in [a_1, b_1]$. Assume that~$n_k$
     and~$[a_k, b_k]$ has been defined for a~$k\ge 1$ so that~$b_k>a_k >0$. Note that there exists
     $M>0$ such that
     \begin{equation}
       \label{eq:union}
       [M, +\infty) \subset \mcup_{n> n_k} n[a_k, b_k].
     \end{equation}
     Take $a \ge  M$ such that~$|f(a)|> \epsilon$. By the continuity of~$f$, there exists
     $b>a$ such that~$|f(x)|>\epsilon$ for all~$x\in[a, b]$. According to~\eqref{eq:union},
     there exists $n_{k+1} > n_k$ such that~$[a, b] \subset n_{k+1}[a_k, b_k]$.
     Let~$a_{k+1} = a/n_{k+1}$ and~$b_{k+1} = b/n_{k+1}$. Then~$b_{k+1}>a_{k+1}>0$, $[a_{k+1},
     b_{k+1}]\subset[a_k, b_k]$, and~$f(n_{k+1}x) > \epsilon$ for all~$x\in [a_{k+1}, b_{k+1}]$.
     This finishes the induction and completes the proof.   

  2. Proof by Baire's Category Theorem. Let~$\epsilon$ be a positive constant. Since~$f(nx)\to 0$
     for all~$x>0$, we have 
  \begin{equation}
    \mcup_{N=1}^\infty \mcap_{n=N}^\infty\left\{ x >0 \mathrel{:} |f(nx)| \le \epsilon \right\}
    \;=\; (0,+\infty).
  \end{equation}
  By the continuity of~$f$, $\bigcap_{n=N}^\infty \{x> 0 \mathrel{:} |f(nx)\le \epsilon|\}$ is closed for
  each~$N \ge 1$.
  As~$\real$ is a Baire space, $(0,+\infty)$ is not meager. Thus there exists~$N$ and an open
  interval~$(a, b)$ such that  
  \begin{equation}
    (a, b) \subset \mcap_{n=N}^\infty\left\{ x >0 \mathrel{:} |f(nx)| \le \epsilon \right\}.
  \end{equation}
  Therefore
  \begin{equation}
    \label{eq:abe}
    n(a, b) \subset \left\{ x >0 \mathrel{:} |f(x)| \le \epsilon \right\}
    \quad \text{ for each } \quad n\ge N.
  \end{equation}
  Note that there exists~$M >0$ such that 
  \begin{equation}
    \label{eq:abm}
    [M, \infty) \subset \mcup_{n=N}^\infty n(a,b). 
  \end{equation}
  Combining~\eqref{eq:abe}--\eqref{eq:abm}, we have
  \begin{equation}
    [M, \infty) \subset \left\{ x >0 \mathrel{:} |f(x)| \le \epsilon \right\}, 
  \end{equation}
  which completes the proof. 

\end{proof}

\begin{proposition}
  Let~$f\mathrel{:} (0,+\infty)\to \real$ be a continuous function. If~$\lim_{n\to \infty}f(nx)$ 
  exists for all~$x>0$, then~$\lim_{x\to +\infty} f(x)$ exists.
\end{proposition}



\small
\bibliography{ref-baire}
\bibliographystyle{plain}

\end{document}
